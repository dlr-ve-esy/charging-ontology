\section{Results}
\label{results}

At the time of this publication the ontology has an open git
repository\footnote{Note for the editors: The publishing of the repository is
being subject to export control, we would update the links accordingly in the
camera ready version. }. This repository has a developer documentation where the
motivating scenarios along their competency questions are written. A quick
introduction for developers is also part of this documentation. This
documentation is written using exclusively markdown files which can be rendered
into webpages with different packages, we opt for using
MkDocs\footnote{https://www.mkdocs.org/} because of its simplicity. For the
ontology documentation and the IRI resolution we are still exploring solutions.
We are tending towards Widoco\footnote{https://github.com/dgarijo/Widoco}.

We implemented a python test suite that allows the incremental development of
the ontology using competency questions. This suite classifies the queries in
two types. Those who act on the class level (TBox) like \hyperref[CQ1.1]{CQ 1.1}
and those that act on the individual level (ABox) like \hyperref[CQ2.0]{CQ 2.0}.
To handle the ABox queries we rewrite the natural language questions as ASK
SPARQL queries that return either true or false values and declare a model in
Turtle Syntax that instantiates the referred classes. For example
\hyperref[CQ2.0]{CQ 2.0} is written as listing \ref{lst:1} and its corresponding
model \ref{lst:2} populates the ABox. The TBox questions are evaluated by
evaluating entailment using DL queries such as the one in listing \ref{lst:3}
which corresponds to \hyperref[CQ1.1]{CQ 1.1}.

\begin{listing}[h]
    
    \begin{minted}[fontsize=\small]{sparql}
        ASK WHERE {
            SELECT (COUNT(?parkingSpaces) AS ?vehicleCapacity)  
            WHERE  {  :SomeParkingArea obo:BFO_0000178 ?parkingSpaces . }
            HAVING ( ?vehicleCapacity = 2 ) }
    \end{minted}
    \caption{Example ABox query. (Given a parking area with two parking places) What is the (vehicle) capacity of parking lot P? (2). The namespaces are omitted.}
    \label{lst:1}
\end{listing}

\begin{listing}[h]
    \begin{minted}[fontsize=\small]{turtle}
        :SomeParkingSpaceA a chio:CHIO_00000002 .
        :SomeParkingSpaceB a chio:CHIO_00000002 .
        :SomeParkingArea a chio:CHIO_00000001 ;
                         obo:BFO_0000178 :SomeParkingSpaceA,
                                         :SomeParkingSpaceB .
    \end{minted}
    \caption{Example ABox instances. Two charging spaces (that can hold at most one car at the time) are part of some parking area.The chio namespace refers to the charging ontology.}
    \label{lst:2}
\end{listing}

\begin{listing}[h]
    \begin{minted}[fontsize=\small]{text}
        'charging station' SubClassOf 'parking facility' 
            and 'has continuant part' some 'charging column'
    \end{minted}
    \caption{Example DL Query used to evaluate TBox competency. A charging station is a kind of parking facility and has charging columns as parts.}
    \label{lst:3}
\end{listing}

We also import the entire CCO version of the BFO which already
includes basic RO axioms. We import  the excerpts from OEO and CCO to be
reutilized using scripts to increase transparency and reproducibility. These
steps point to specific versions of the ontologies that can be updated based on
developing needs. 

The next milestone is to have implemented the terminology associated with the
first two scenarios. Followed by a demonstration application based on scenario
one using the FAIR version of the German network agency data that we extracted
for \cite{ArellanoRuiz.2024}.

One important open issue is hosting of the ontology. Currently, we use the
namespace of the Open Energy Platform, but we are not sure if they would
consider it part of the scope of their project. One alternative could be
upcoming repository of ontologies often mentioned in the discussion board of
the CCO. At the time of writing this paper we propose having this discussion
during the workshops of these proceedings.




    