\section{Existing ontology work}
\label{existingontologies}

One of the key aspects of FAIR ontologies is that they are reusable and profit
from the reusability of existing ontologies \cite{PovedaVillalon.2020}. We did
an extensive analysis of existing ontologies with infrastructure, charging
stations, electric vehicles, and electricity grid in scope. We also rely
strongly on the review performed by Katsumi and Fox regarding transport planning
vocabularies \cite{Katsumi.2018}. In this section, we summarize the ontologies
we considered for reutilization and justify the situations in which we decide
not to reutilize existing concepts.


\subsection{The Open Energy Ontology}

The OEO has been in active development since 2021 with several releases since
then. It exists to address a technical gap associated with knowledge management
in the field of energy systems analysis. Said gap is the lack of common
semantics to annotate and share datasets and tools associated with the mentioned
discipline. The ontology is part of a larger data ecosystem called the Open
Energy Family (OEF), which allows researchers to share data sources and results
following the FAIR principles. It has had moderate success, particularly within
the context of projects associated with the OEF such as the Open Energy Platform
(OEP) \cite{Hulk.2024}. One of the characteristics that make this and other FAIR
ontologies transparent and accessible is the fact that it is being openly
developed in a shared
repository\footnote{https://github.com/OpenEnergyPlatform/ontology}. An
important work on the inclusion of concepts coming from the transport sector was
performed by Mittermeier \cite{Mittermeier.2023}, who did several
implementations associated with the topic. However, since the size of the task
is larger than what can be achieved during a master thesis, many implementations
were left open in the form of GitHub issues. In some of these issues, it was
made clear that the scope of the OEO is in some cases beyond what is often
necessary to represent phenomena in the transport sector.


In our charging infrastructure ontology, we need a way of describing vehicles,
particularly electric vehicles. The OEO has a rich taxonomy of vehicles that
rest on the definition of artificial objects which are in the context of BFO
`causally unified material entities deliberately manufactured by humans to
address a particular purpose'. The taxonomy has two parallel ramifications, one
associated with its energy consumption mode like `electric vehicle', `internal
combustion vehicle', and `gas turbine vehicle' and another associated with its
operational medium such as `land vehicle', `aircraft' or `watercraft'. The
former has axiomatization significant for electric grid and energy systems
models which can be seen in figure \ref{oeoev:b} the latter has no own axioms
and relies mainly on the former. For our application is only interesting to use
the taxonomy of electric vehicles, which should include plug-in hybrid electric
vehicles (Figure \ref{oeoev:a}). Its land vehicle taxonomy is rich (figure
\ref{landvehicletaxoeo}) and contains elements that might produce conflicts with
any future implementation in a transport ontology. Because of this, we rely on
the Common Core Ontologies (CCO) for a lighter vehicle taxonomy, this will be
clarified in its respective section.

\begin{figure}
    \centering
    \subfigure[OEO electric vehicle taxonomy.]{\label{oeoev:a}\includegraphics{images/OEOVehicles.pdf}}
    \subfigure[OEO electric vehicle commitments]{\label{oeoev:b}\includegraphics{images/OEOEV.pdf}}
    \caption{OEO ontological commitments relevant to the context of charging infrastructure. Note that we exclude grid related commitments, mostly because we have to properly outline a scenario to implement them. Solid arrows represent super-class relations.}
\end{figure}

\subsection{The Common Core Ontologies}

The CCO are twelve mid-level ontologies built as an extension of BFO and the
Relations Ontology (RO) intended to be used as a basis to model domains of
interest such as transportation infrastructure and spacecraft
\cite{Rudnicki.23September2020}. Like BFO, it is a realist ontology, which
means that it intends to model the entities data represent. The ontology
avoids being prescriptive and instead lets data modelers decide which asserted
class axioms are relevant for their particular applications. It also comes with
a guide that lets non-ontology experts understand how to implement terms which
is helpful to involve domain experts of other fields in ontology development.

We import a small subset of terms coming from these ontologies. Indeed, we avoid
importing whole ontologies by opting to do punctual extractions. This was
decided to avoid the convolution of the final product since we are expecting
developers to work with the ontology software Protégé. If we imported all the
classes they would have a harder time finding where to place their new terms. We
extract multiple taxonomies and some axioms from these ontologies. These are
explained in the rest of this section.

\subsubsection{Vehicle taxonomy}

From the `Artifact Ontology' we extract terminology associated with vehicles. We
consider that they offer a more manageable and expandable taxonomy of vehicles
that can be utilized across different fields of application. The OEO classifies
vehicles based on their energy consumption which is practical for their
applications but, since we are not interested in non-electric vehicles, these
become superfluous. Another reason to use the `Artifact Ontology' axiomatization
is to profit from the other declarations coming from the same suite of
ontologies. The top upper levels of both taxonomies are slightly different
besides both using BFO. The OEO classifies vehicles as `artificial objects'
whereas the CCO as `material artifacts'. The lower levels, despite being
developed independently, are so similar that they are practically already
interoperable. The taxonomies can be compared by looking at figures
\ref{ccovectax} and \ref{landvehicletaxoeo}.

\begin{figure}[h]
    \centering
    \includegraphics{images/CCOVehicles.pdf}
    \caption{The CCO vehicle taxonomy is slim and has fewer asserted axioms. Solid arrows represent super-class relations.}
    \label{ccovectax}
\end{figure}
\begin{figure}[h]
    
    \centering
    \includegraphics{images/OEOLVehicles.pdf}
    \caption{The OEO Land vehicle taxonomy is rich in axioms associated with the vehicles' energy consumption. Solid arrows represent super-class relations.}
    \label{landvehicletaxoeo}
\end{figure}

\subsubsection{Infrastructure}

The CCO offers a construct that aids in classifying `material artifacts' as
infrastructure. It uses what in BFO are called `roles'. This allows arbitrary
assignment of the class. This is practical to us because charging stations in
reality are not always infrastructure, they are only in virtue of the agents who
assign them this role. In this sense, a home wall box is not infrastructure for
a government agency but a public column is. Infrastructure rarely comes as units
but as complex aggregates of `artifacts', this is the way we opt to import the
concept of `infrastructure system'. These axioms can be visualized in figure
\ref{infrastructurefigs}.

\begin{figure}[h]
    \centering
    \includegraphics{images/infrastructureSystem.pdf}
    \caption{Common Core Ontologies infrastructure constraints. Solid arrows represent super-class relations}
    \label{infrastructurefigs} 
\end{figure}

\subsubsection{Other imports}

We also import terms from the facility, information entity, and geospatial
ontologies. We use the facility ontology to handle concepts like parking lots
and dedicated charging stations. An alternative would be to stick to using
material entities, but we consider this differentiation practical in the long
run. The information entity ontology provides us with the `designates' relation
which the CCO recommends using to designate geospatial data to entities. The
geospatial ontology provides terms like city and continent which we use to
manage addresses.


\subsection{iCity Parking ontology}

The TPSO has a module for terminology associated with parking which provides
concepts like parking spaces, areas, and fees. It also offers axioms for
charging stations, but these are too shallow for our applications as they are at
most features of parking spaces. The subclassification of charging stations are
`standard', `medium', and `fast' which are based on the definitions of some
`Environmental Protection Department' whose source was not explicitly pointed.
Whether having such a classification is meaningful to our applications is yet to
be defined. The TPSO has its own top-level ontology modules which supply axiom
definitions for change, mereology, and time. These modules are incompatible with
the BFO. The Change module of the ontology relies on the utilization of a
four-dimensional approach to model time-changing concepts. This means that every
object has a perdurant and its manifestations bear their changes. Since we are
not intending to axiomatize time relations in OWL and instead delegate that to
data modelers we opt not to share that approach. BFO handles descriptions of
change differently, this will be addressed in its respective section. Some
axioms that are interesting to us from the TPSO Parking ontology, excluding
mereology of parking areas, can be visualized in figure \ref{parkingfig}. Our
approach to reutilize them consists of doing an implementation using BFO and
then adding mappings to this ontology. 

\begin{figure}[h]
    \centering
    \includegraphics{images/PARKING.pdf}
    \caption{iCity parking ontology commitments associated with charging infrastructure. Solid arrows represent super-class relations.}
    \label{parkingfig}
\end{figure}

\subsection{The Basic Formal Ontology}
\label{upperlevel}

Top-level ontologies or foundational ontologies intend to model domain-neutral
categories and relations \cite{Arp.2015}. These are not a hard requirement to
build an ontology but can make a difference regarding its interoperability.
Choosing (or developing) a top-level ontology is not an easy task because a
modeler needs to have a deep understanding not only of their domain and the
possible applications but also any possible adjacent domains that may operate
with the same data. In 2022 a special issue of applied ontology
\cite{Borgo.2022} allowed expert users and developers to exemplify the usage of
multiple prominent top-level ontologies. In theory, we would select a top-level
ontology using these examples as a way to implement the motivating scenarios
defined in section \ref{methodology}. But since we intend to build on the
efforts from the OEO we streamlined to using BFO. This decision is not without
compromises. 

BFO sacrifices expressivity for a simpler modeling intuition. It has weaknesses
in the field of modal propositions which are prominent in both transportation
research en energy systems analysis, particularly in the context of forecasts
and simulation\footnote{However this weakness is addressed by CCO with their
ModalRelationOntology which we consider importing. }. Unlike the TPSO
foundational modules, BFO delegates its time indexing to the applications by
keeping it outside the OWL implementation. The perdurant component of entities
is handled by pointing to their `history', a practical and versatile construct.
BFO has a rather simple initial learning curve that can be used to facilitate
the inclusion of new developers. But still becomes steep when dealing with some
of its more complex terms (i.e. `process profile', `disposition' vs `quality'),
this can lead to frustrations and misuses.

There are some important features of BFO that we intend to take advantage of.
The first is that it allows the existence of `sites' in the same dimension as
`material entities'. This is because charging events depend on the dynamics of
vehicles being present and absent. We also need these entities to characterize
the mereotopology of charging stations which usually consider parking spaces as
their parts which in place can be part of parking lots along other parking
spaces. To address this, we will rely on reinterpreting the axioms from the TPSO
in the context of BFO, in a future stage we intend to explore the possibility of
unintended models (i.e. a vehicle becoming infrastructure). A second important
feature is the concept of `process profiles' since we consider that the charging
event can have multiple dimensions that can describe it that have to be
associated with each other. Using those we can describe events by virtue of
their occupancy, power rates, and energy transfers among other characteristics.
