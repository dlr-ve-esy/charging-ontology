\section{Conclusion}
\label{conclusion}

Questions on electrification of the transport sector and other research trends
have as the ultimate goal to put a stop to climate-change driving carbon
emissions. These study fields deal with complex systems whose boundaries tend to
differ based on the research questions themselves. These differences become
troublesome when exchanging data because, in most cases, the person producing it
is not always available to clarify misunderstandings. Metadata improves this
situation significantly, but their content is also subject to interpretation.
While ontologies can't solve problems of ambiguity, they can ease them by
offering a way of explicitly declaring semantics. With this small ontology
project, we intend to offer a reference point for communication not only between
transport and energy researchers but also any person whose work involves
electric vehicle charging stations which are components of larger infrastructure
systems. One of the most important lessons we learned during this work is that
reutilization is the most valuable tool we have when developing FAIR ontologies.
If one looks long enough, there are already some brilliant minds who meditated
on the same or similar research questions. The ontology development is
streamlined thanks to the effort that we invested in testing infrastructure. We
consider that the methodology we used to prepare the environment can be
re-utilized for other ontologies. We can easily support this statement because
an important part of what we did was applying what we learned from the OEO
development. In further works, we expect to use our approach to create, map, and
expand ontologies usable for transportation research and cover other niches that
are out of the scope of infrastructure ontologies like the OEO, the CCO, and the
TPSO.