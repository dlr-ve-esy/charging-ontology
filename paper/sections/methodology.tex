\section{Methodology}
\label{methodology}

The ontology development loop is inspired by the open collaborative development
of the OEO developing team. This consists on having user needs, stated in
public issues, as primary driving force of ontology development. This was not
decided based on some detailed evaluation of ontology development methods but
out of experience and satisfactory results that manifest in the OEO relatively
consistent release cycle. The main difference consists of a stricter approach
in regard to new terminology coming into the ontology. Instead of implementing
single terms per request, we will expect users and developers to either adjust
their proposed terms to existing applications or to propose a new complete
motivating scenario. This is to overcome scoping issues that often arise during
the development of OEO. 


\subsection{Motivating Scenarios}

Motivating scenarios are descriptions of potential real or theoretical
applications of the ontology which comprise the kind of questions that the
ontology is intended to address.  They were first proposed as a method used for
the Toronto Virtual Enterprise (TOVE) ontology development
\cite{Gruninger.1995}. We take inspiration on the original approach but adapt
it to a modern context where we take advantage of ticket systems and public
version control based development. Competency questions are the smallest
component of a scenario and in our development approach they are first-class
citizens. The rest of this section will show the first motivating scenarios
considered for this ontology along some exemplary competency questions. Notice
that some competency questions are written as statements which can be used as
true/false queries in SPARQL or as an entailment test with a reasoner. The
numeration of the competency questions is not sequential, this is intended as
they are excerpts of the full list of questions which can be found in the
documentation of the ontology\footnote{Note for the editors: The publishing of
the documentation is being subject to export control, we would update the links
accordingly in the camera ready version. }

\subsubsection*{Scenario 1: The charging infrastructure register}

This scenario is heavily inspired by the German charging infrastructure
register \cite{Bundesnetzagentur.27Oct2023}, and it probably captures most of
the requirements of this ontology. The scenario covers terminology and axioms
necessary to perform descriptions in regard to where infrastructure is found,
which power they are able to deliver and what kind of connector they have.

\begin{namedbreak}[Competency question 1.0]
    A public charging station is a kind of transportation infrastructure.
\end{namedbreak}

\label{CQ1.1}\begin{namedbreak}[Competency question 1.1]
    A charging station has charging columns as parts that can change during its lifetime.
\end{namedbreak}
\begin{namedbreak}[Competency question 1.3]
    A charging point is compatible with some plugs.
\end{namedbreak}

\begin{namedbreak}[Competency question 1.5]
    A charging station has commissioning and decommissioning dates which delimit its lifetime.
\end{namedbreak}

\subsubsection*{Scenario 2: iCity Project Smart Parking Applications}

This scenario is an extract of the iCity project by Katsumi and Fox.
Particularly the section smart parking applications \cite{Katsumi.2020}. These
are selected examples of the subset of questions relevant to us. They are
interesting because charging infrastructure is intimately connected with
parking infrastructure. These queries rely on a lot in geographical queries and
may overlap with scenario 1, but they have a perspective more in line with
daily operation of the stations. For more details on the ontology visit its
repository\footnote{https://github.com/EnterpriseIntegrationLab/icity}.


\label{CQ2.0}\begin{namedbreak}[Competency question 2.0]
    What is the address of the parking lot P?
\end{namedbreak}

\begin{namedbreak}[Competency question 2.5]
    Is a particular parking lot open to the public at a given time?
\end{namedbreak}

\begin{namedbreak}[Competency question 2.6]
    How many parking spots are designated for electric vehicles in a particular parking lot?
\end{namedbreak}

\begin{namedbreak}[Competency question 2.7]
    What types of electric vehicle chargers are available in a particular parking lot?
\end{namedbreak}