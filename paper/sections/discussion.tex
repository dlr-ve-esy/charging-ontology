\section{Discussion}
\label{discussion}

One of the main disadvantages of writing competency questions as queries for
most concepts is that implementations are slow. We consider that the benefits
outweigh the drawbacks as the outputs are ensured to be concise without further
validation. Sometimes is challenging to come up with competency questions when
there are no implementations to refer to. The CCO provides a clear documentation
that we can turn to in such cases.

The next milestone is to complete the terminology associated with the first two
scenarios. As well as designing two further scenarios, each associated with a
service. The first consists of a day-ahead stochastic forecasting application
for individual and groups of stations. The second is for the aggregated
bidirectional charging potentials in future energy systems.  
We intend to do practical demonstrations  by adjusting data models based on
scenario one using the FAIR version of the German network agency data that we
extracted for \cite{ArellanoRuiz.2024}.

One important open issue is hosting of the ontology. Currently, we use the
namespace of the Open Energy Platform, but we are not sure if they would
consider it part of the scope of their project. One alternative could be
upcoming repository of ontologies often mentioned in the discussion board of the
CCO. Our last alternative is to prepare some form of self-hosting. At the time
of writing this paper we propose having this discussion during the workshops of
these proceedings.
