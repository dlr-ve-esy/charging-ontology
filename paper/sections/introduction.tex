\section{Introduction}
\label{introduction}
Energy systems analysis is a discipline where specialized software tools are
used to study the development of possible future energy systems. One modern
example of such studies is \cite{Victoria.2022} which uses a European instance
of the energy systems optimization model PyPSA \cite{Brown.2018} to evaluate
decarbonization pathways. In such publications, particularly those associated to
German contexts, the concept of `sector-coupling' is often used to state that
the analysis comprises the evaluation of phenomena in multiple energy consuming
and producing sectors \cite{Fridgen.2020}. This concept is prone to ambiguity.
Depending on the publication it may refer to the active process of replacing
fossil energy consuming processes with renewable alternatives, the integration
of energy consuming and producing sector infrastructure or simply the
co-analysis of two or more energy consuming or producing sectors. In general the
concept tends to encompass some sort of interaction between energy consumers and
producers. \cite{Ramsebner.2021} provides an extensive analysis of the concept
but does not propose an ultimate definition, instead they propose their own
interpretation\footnote{The discussion on how to axiomatize a concept such as
sector-coupling is beyond the scope of this paper, but it can be followed at:
https://github.com/OpenEnergyPlatform/ontology/issues/1521}. The
transport-energy nexus is often a subject on such sector-coupling discussions,
for example \cite{Robinius.2017} reviews multiple studies where transport and
power are co-analysed and interprets them based on their chosen definition of
sector-coupling. The difficulty of concretizing such a term comes from the fact
that these disciplines deal with systems like electricity grids and transport
networks which are instances of complex systems. One of the first barriers of
co-analysing  such systems lies on the terminology used in the disciplines
studying them. It is hard to properly interpret such sector-coupling analyses
without a solid common semantic basis. To address this and other similar
problems the Open Energy Ontology (OEO) \cite{Booshehri.2021} was conceived, but
once the analyses go too deeply into the field of transport research its
usability dwindles. There are many points of interaction between an energy
system and a transport system, the latter which accounts globally for around 8
Gton of CO\textsubscript{2} \cite{IEA.2023}. Of these points of interaction, one
of the most relevant is the battery electric vehicle charging infrastructure,
because it is directly associated with the process of electrification of the
fulfillment of transport needs. In this paper, we propose an ontology as an
initial interoperability layer between energy systems analysis and
transportation research that allows to represent concepts associated to the
process of charging an electric vehicle from the point of view of a grid and a
transport network. The rest of this paper is divided in six sections. Section
\ref{statementofneed} consists of the justification, philosophy and approach for
the designing of an ontology to allow the description of concepts and phenomena
associated to charging of battery electric vehicles. Section
\ref{existingontologies} describes what has been done so far in context of other
ontologies to describe concepts of transportation research, during this section
the shortcoming and advantages of utilizing the Basic Formal Ontology (BFO)
\cite{Arp.2015} as an top-level ontology. Section \ref{methodology} describes
the concrete needs, proposed methodology, tools and competency questions.
Section \ref{results} will be a description of what has been done so far.
Section \ref{discussion} points out the challenges and provides a description of
the future roadmap. And we conclude our paper in section \ref{conclusion}.

