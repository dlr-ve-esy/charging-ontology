\section{Introduction}

Specialized software tools have often been used for the analysis of development
of possible future energy systems. One modern example of such studies is
\cite{Victoria.2022} which uses a European instance of the model PyPSA
\cite{Brown.2018} to evaluate decarbonization pathways. In such publications,
particularly those associated to German contexts, the concept of
`sector-coupling' is often used to state that the analysis comprises the
evaluation of phenomena in multiple energy consuming and producing sectors
\cite{Fridgen.2020}. This concept is however convoluted since, depending on the
publication it may refer to the active process of replacing fossil energy
consuming processes with renewable alternatives, the integration of energy
consuming and producing sector infrastructure or simply the co-analysis of two
or more energy consuming or producing sectors. In general the concept tends to
encompass some sort of interaction between energy consumers and producers,
\cite{Ramsebner.2021} provides an extensive analysis of the concept but fails
to propose an ultimate definition, instead they propose their own
interpretation\footnote{The discussion on how to axiomatize a concept such as
sector-coupling is beyond the scope of this paper, but it can be followed at:
https://github.com/OpenEnergyPlatform/ontology/issues/1521}. The
transport-energy nexus is often a subject of discussion on such sector-coupling
discussion, for example \cite{Robinius.2017} reviews multiple studies where
transport and power are co-analysed and interprets them based on their chosen
definition of sector-coupling. The co-analysis complex systems such as the
power and transport systems is a complex subject and one of the first barriers
lies on the terminology used in both fields. Is hard to properly interpret such
sector-coupling analyses without a solid common semantic basis, to address this
and other similar problems the Open Energy Ontology (OEO) \cite{Booshehri.2021}
was conceived, but once the analyses go to deeply into the field of transport
research its usability dwindles. There are many points of interaction between
an energy system and a transport system, the latter which, according to
\cite{IEA.2023}, account globally for around 8 Gigatonnes of CO2. One of the
most relevant points of interaction is the electric vehicle charging
infrastructure, because it is directly associated with the process of
electrification of the fulfilment of transport needs. In this paper, we propose
an ontology as an initial interoperability layer between energy systems
analysis and transport research that allows to represent concepts associated to
the process of charging an electric vehicle from the point of view of a grid
and a transport network. This rest paper is divided in six sections. The second
section consists of the justification, philosophy and approach for the designing of an ontology that
complements the OEO to allow the description of concepts and phenomena
associated to charging of battery electric vehicles. The third will
describe what has been done so far in context of the OEO to describe concepts
of transport research. The fourth is a review of existing ontologies and
vocabularies covering our topic.  The fifth will describe the concrete needs,
proposed methodology, tools and competency questions, during this section the
shortcoming and advantages of utilizing the Basic Formal Ontology (BFO)
\cite{Arp.2015} as an upper level ontology. The sixth section will be a
description of what has been done so far. And the last section is a conclusion
and description of the future roadmap.

