\section{Introduction}
\label{introduction}
Energy systems analysis is a discipline where specialized software tools are
used to study the development of possible future energy systems. One modern
example of such studies is \cite{Victoria.2022} which uses a European instance
of the energy systems optimization model PyPSA \cite{Brown.2018} to evaluate
decarbonization pathways. In such publications, particularly those associated
with German contexts, the concept of `sector coupling' is often used to state
that the analysis comprises the evaluation of phenomena in multiple
energy-consuming and producing sectors \cite{Fridgen.2020}. This concept is
prone to ambiguity. Depending on the publication it may refer to the active
process of replacing fossil energy-consuming processes with renewable
alternatives, the integration of energy-consuming and producing sector
infrastructure, or simply the co-analysis of two or more energy-consuming or
producing sectors. In general, the concept tends to encompass some sort of
interaction between energy consumers and producers. \cite{Ramsebner.2021}
provides an extensive analysis of the concept but does not propose an ultimate
definition, instead, they propose their interpretation\footnote{The discussion
on how to axiomatize a concept such as sector-coupling is beyond the scope of
this paper, but it can be followed at:
https://github.com/OpenEnergyPlatform/ontology/issues/1521}. The
transport-energy nexus is often a subject of such sector-coupling discussions.
For example, \cite{Robinius.2017} reviews multiple studies where transport and
power are co-analyzed and interprets them based on their chosen definition of
sector coupling. The difficulty of concretizing such a term comes from the fact
that these disciplines deal with systems like electricity grids and transport
networks which are instances of complex systems. One of the first barriers to
co-analyzing such systems lies in the terminology used in the disciplines
studying them. It is hard to properly interpret such sector coupling analyses
without a solid common semantic basis. To address this and other similar
problems the Open Energy Ontology (OEO) \cite{Booshehri.2021} was conceived, but
once the analyses go too deeply into the field of transport research its
usability dwindles. There are many points of interaction between an energy
system and a transport system. One of the most relevant is the battery electric
vehicle charging infrastructure because it is directly associated with the
process of electrification of the fulfillment of transport needs which accounted
globally in 2018 for around 8 Gton of CO\textsubscript{2} emissions
\cite{IEA.2023}. In this paper, we propose an ontology as an initial
interoperability layer between energy systems analysis and transportation
research that allows us to represent concepts associated with the process of
charging an electric vehicle from the point of view of a grid and a transport
network. The rest of this paper is divided into six sections. Section
\ref{statementofneed} consists of the justification, philosophy, and approach for
the designing of an ontology that allows the description of concepts and
phenomena associated with the charging of battery electric vehicles. Section
\ref{existingontologies} is a summary of an extensive review of existing
ontologies dealing with the topic, during this section the shortcomings and
advantages of utilizing the Basic Formal Ontology (BFO) \cite{Arp.2015} as a
top-level ontology. Section \ref{methodology} describes the concrete needs,
proposed methodology and competency questions. Section \ref{results} will be a
description of current achievements and implementations. Section
\ref{discussion} points out the challenges and describes the future roadmap. We
conclude our paper in section \ref{conclusion}.

