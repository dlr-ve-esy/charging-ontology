\section{Background}
\label{statementofneed}
The need for an ontology that represents entities and phenomena associated with
charging infrastructure was identified first during the process of implementing
the FAIR principles on the data published by the German Federal Network Agency
\cite{ArellanoRuiz.2024}. There we noticed that the existing ontologies and
vocabularies were insufficient to annotate concepts like charging plug type.
After a short discussion in the OEO issue
board\footnote{https://github.com/OpenEnergyPlatform/ontology/issues/1597} we
concluded that things like `power plug' and potential subclasses of
the same are beyond the scope of the ontology. Along with concepts like `parking
place' or `mobility mode', which are often used during transportation research
analyses. It has been hard to argue to include them in the OEO, despite them
being used during cross-domain analyses like the one done by \cite{Hecht.2022}.
This need is deeply explored by \cite{Mittermeier.2023}, but their approach
used to address it relied mostly on including the terms in the OEO which we
consider not viable in the long run. However, their research provides a solid
theoretical basis for further developments in the field of knowledge
representation of terminology in the discipline of transport research.

Katsumi and Fox \cite{Katsumi.2018} provide an extensive review of transport
ontologies, taxonomies, vocabularies, and other tools for data representing
information on transport systems. They shed light on the large complexity of
the data environments associated with this discipline. Their research concretized
in the creation of the iCity Transportation Planning Suite of Ontologies (TPSO)
\cite{Katsumi.2019}. They touch on the topic of charging infrastructure on a
superficial level which is sufficient for transportation planning tasks. They
leave the topic of adding more detail to BEV charging to future work. What we
propose is an extension that not only deals with transportation planning but
also promises interoperability with planning grid infrastructure and energy
consumption estimations. The particular applications are explored in section
\ref{methodology} where we define scenarios that drive and bind the ontology
development. It is important to point out that the actual implementations
differ significantly from the TPSO because they define their own top-level
ontology which is fundamentally different from BFO, the particularities of
these differences are described in section \ref{upperlevel}.

There are some other vocabularies and models that deal with charging
infrastructure, but most of them are constrained by their specific applications,
as they prescribe what would be expected from the data models produced out of
them. One important mention is the work from \cite{MaximeLefrancois.2017} who
proposed a power-systems ontology aimed at interoperability of the Internet of
Things (IoT) domain. They provide a rich axiomatization of charging
infrastructure specialized towards power systems' representation. Some elements
of interest to us are present, but most relations are too specific to IoT.
Relations with entities external to an IoT network like vehicles are not
considered and adding them to the imported ontology would render it inconsistent
given its rigid range and domain constraints. 