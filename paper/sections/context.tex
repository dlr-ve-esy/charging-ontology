\section{Background}
\label{statementofneed}
The need of an ontology that represents entities and phenomena associated with
charging infrastructure was identified first during the process of implementing
the FAIR principles on the data published by the German Network Agency
\cite{ArellanoRuiz.2024}. There we noticed that the existing ontologies and
vocabularies where insufficient to annotate concepts like charging plug type.
After a short discussion in the OEO development
forum\footnote{https://github.com/OpenEnergyPlatform/ontology/issues/1597} we
came to the conclusion that thing like `power plug' and potential subclasses of
the same are beyond the scope of the ontology. Along concepts like `parking
place' or `mobility mode', which are concepts that are often used during
transportation research analyses, it has been hard to argue to include them in the
ontology, despite them being used during cross-domain analyses like
\cite{Hecht.2022}. This need is deeply explored by \cite{Mittermeier.2023}, but
their approach used to address it relied mostly on including the terms in the
OEO which we consider not viable in the long run. However, their research
provides a solid theoretical basis to further developments in the field of
knowledge representation of terminology in the discipline of transport
research.

Katsumi and Fox \cite{Katsumi.2018} provide an extensive review of transport
ontologies, taxonomies, vocabularies and other tools for data representing
information of transport systems. They shed light on the large complexity of
the data environments associated to this discipline. Their research concretized
in the creation of the iCity Transportation Planning Suite of Ontologies (TPSO)
\cite{Katsumi.2019}. We will be referencing their work in further sections.
They touch on the topic of charging infrastructure on a superficial level which
is sufficient for transportation planning tasks. What we propose in this work
is an extension that promises interoperability with planning grid
infrastructure and energy consumption estimations. The particular applications
are explored in section \ref{methodology} where we define scenarios that drive
and bound the ontology development. It is important to point out that the actual
implementations differ significantly from the TPSO because they define their
own upper level ontology which is fundamentally different from BFO, the
particularities of these differences are described in section \ref{upperlevel}.