\begin{abstract}
Multidiscipline analyses are prone to misunderstandings associated with rich
discipline-specific semantics. The research of low carbon energy systems is an
example of such applications where concepts coming from multiple expert groups
collide. One very relevant example is the transport-energy nexus where
`sector-coupling' analyses are performed. In this context battery electric
vehicle (BEV) charging infrastructure comprises a main point of interaction for
both research disciplines. Ontologies like the Open Energy Ontology (OEO) where
conceived to aid in the concretization of agreement in such multidisciplinary
research. But since it is a tool whose focus is  energy systems research, it falls
short in concepts associated with transport research. In this paper we propose
a FAIR Ontology, which should work as a first interoperability layer between
the OEO and other ontologies intending to represent concepts associated with
BEV Charging Infrastructure. We develop this ontology using a methodology
inspired by the OEO with a more strict requirements engineering approach. This
methodology relies strongly on motivating scenarios and competency questions
to keep the ontology slim. Current achievements are a documented development
environment and reutilization of existing ontology terms coming from ontologies
like the OEO, the Common Core Ontologies (CCO) and the iCity Transport Planning
Suite of Ontologies (TPSO).
\end{abstract}